%%=============================================================================
%% Discussie
%%=============================================================================

\chapter{\IfLanguageName{dutch}{Discussie}{Discussion}}%
\label{ch:discussie}

In dit hoofdstuk worden de resultaten geïnterpreteerd in het licht van de onderzoeksvragen. Waar in het vorige hoofdstuk de nadruk lag op objectieve rapportering, staat hier de duiding en kritische reflectie centraal.

\section{Interpretatie van modelprestaties}

Eerst wordt besproken in welke mate de ontwikkelde modellen erin slagen om wedstrijdstatistieken accuraat te voorspellen. Hierbij wordt nagegaan:
\begin{itemize}
    \item of de behaalde prestaties praktisch relevant zijn,
    \item welke doelvariabelen beter voorspelbaar blijken dan andere,
    \item welke beperkingen zichtbaar worden in termen van generaliseerbaarheid.
\end{itemize}

Daarnaast wordt gereflecteerd op de keuze van de modellen en de impact van feature-engineering op de uiteindelijke prestaties.

\section{Analyse van bepalende factoren}

Vervolgens wordt dieper ingegaan op de factoren die door de modellen als belangrijk werden geïdentificeerd. Er wordt besproken:
\begin{itemize}
    \item welke variabelen consistent een sterke invloed vertonen,
    \item of deze overeenkomen met theoretische verwachtingen,
    \item in welke mate contextuele factoren (zoals locatie of vorm) een significante rol spelen.
\end{itemize}

Hier wordt ook gereflecteerd op mogelijke verklaringen voor onverwachte of contra-intuïtieve bevindingen.

\section{Vergelijking tussen modellen en menselijke expertise}

Een centraal onderdeel van deze bachelorproef is de vergelijking tussen modelverklaringen en menselijke redeneringen. In deze sectie wordt geanalyseerd:
\begin{itemize}
    \item in welke mate experts en modellen dezelfde factoren prioritair achten,
    \item waar significante verschillen optreden,
    \item welke implicaties deze verschillen hebben voor besluitvorming binnen professionele voetbalcontexten.
\end{itemize}

Er wordt nagegaan of modellen bepaalde patronen detecteren die minder expliciet aanwezig zijn in menselijke inschattingen, dan wel of experts contextuele nuances meenemen die moeilijk kwantificeerbaar zijn.

\section{Beperkingen van het onderzoek}

Dit onderzoek kent enkele beperkingen, zoals:
\begin{itemize}
    \item de omvang en kwaliteit van de beschikbare data,
    \item mogelijke vertekeningen in expertselectie,
    \item beperkingen inherent aan de gekozen modellen.
\end{itemize}

Deze beperkingen worden expliciet benoemd om de geldigheid en toepasbaarheid van de conclusies correct te kaderen.

\section{Implicaties en aanbevelingen}

Tot slot worden de implicaties van de bevindingen besproken voor:
\begin{itemize}
    \item professionele voetbalclubs,
    \item data-analisten binnen sportorganisaties,
    \item toekomstig onderzoek binnen sportvoorspellingen en interpreteerbare machine learning.
\end{itemize}

Hier worden concrete aanbevelingen geformuleerd op basis van de verkregen inzichten.