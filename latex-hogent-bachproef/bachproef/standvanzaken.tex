\chapter{\IfLanguageName{dutch}{Stand van zaken}{State of the art}}%
\label{ch:stand-van-zaken}

% Tip: Begin elk hoofdstuk met een paragraaf inleiding die beschrijft hoe
% dit hoofdstuk past binnen het geheel van de bachelorproef. Geef in het
% bijzonder aan wat de link is met het vorige en volgende hoofdstuk.

% Pas na deze inleidende paragraaf komt de eerste sectiehoofding.



In het vorige hoofdstuk zijn de probleemstelling, onderzoeksvragen en doel van deze bachelorproef besproken. 
Om al enige kennis op te doen over deze vragen, wordt in dit hoofdstuk een overzicht gegeven 
van relevante wetenschappelijke literatuur 
rond voetbalvoorspellingen, datanalyse en machine learning. 
Er wordt ingelezen op de gebruikte modellen, bepalende factoren en recente ontwikkelingen in sport-analyses. 
Deze inzichten vormen vervolgens de basis voor de methodologische keuzes die in het volgende hoofdstuk besproken worden.

Tegenwoordig wordt in studies over voetbal voorspellingen data-analyse en machine learning steeds vaker ingezet. 
\textcite{Rodrigues2022} levert een waardevolle bijdrage door een aantal modellen naast elkaar te zetten voor de voorspelling van wedstrijduitslagen in Europese competities. 
Wat zij aantonen, is dat met name classificatiemodellen als Random Forest en Gradient Boosting sterk scoren, 
mits er voldoende historische wedstrijd- en teamstatistieken beschikbaar zijn. 
Daarbij is volgens hen vooral de nadruk op featureselectie om storende ruis en overfitting te beperken, sterk onderbelicht.

\vspace{5pt}

\textcite{Choi2023} gingen op zoek naar diverse machine-learning methoden, waaronder Support Vector Machines, k-Nearest Neighbours en neurale netwerken. 
Zij stelden vast dat de prestaties van modellen sterk afhangen van de kwaliteit van de ingevoerde variabelen en de balans tussen winst-, verlies- en gelijkspeldata. 
Met hun onderzoek benadrukken zij dat een model zelden overal het beste is, waardoor hybride modellen of ensembles vaak betere voorspellingen geven.

\vspace{5pt}

In recent onderzoek gaat \textcite{Sun2025} nog een stap verder door gebruik te maken van quantum neural networks (QNN). 
Deze technologie, gebaseerd op principes uit quantum computing, bood in hun onderzoek een licht betere voorspellingsnauwkeurigheid dan klassieke deep-learningmodellen. 
Hoewel QNN nog in een experimentele fase zit, toont het onderzoek aan dat innovatieve netwerkarchitecturen potentieel hebben binnen sports analytics.

\vspace{5pt}

Verder wordt naast het modelleren van voetbalprestaties ook meer onderzoek gedaan naar de achterliggende factoren. 
\textcite{Fan2023} bestudeerden data van verschillende FIFA-wereldbekers en kwamen tot de conclusie dat factoren als ervaring, 
economie en de kwaliteit van de tegenstander bepalend zijn. We moeten hierbij wel aangeven dat de studie zich voornamelijk richt op internationale toernooien. 
Het laat echter wel zien dat context zeer bepalend kan zijn voor prestaties in sport.

\vspace{5pt}

\textcite{Zhong2024} bestudeerden de impact van weersomstandigheden op de technische prestaties van teams. 
Daarbij stelden ze vast dat factoren als temperatuur, wind en luchtvochtigheid een significante invloed hadden op de nauwkeurigheid van passes, 
de intensiteit en de foutenmarge. 
Hiermee laten zij zien dat externe omgevingsfactoren ertoe doen bij het in kaart brengen van prestatiedata.

\vspace{5pt}

Tot slot lieten \textcite{Wong2025} zien hoe zij een geavanceerd voorspellend framework bouwden waarin verschillende machine-learningmodellen worden samengevoegd
 om sportuitslagen te voorspellen. De focus van hun framework ligt op feature engineering, datareikwijdte en modelvergelijking. 
 Ze bewijzen dat ensemble-technieken, zoals blending en stacking, doorgaans betere prestaties leveren dan losse modellen.

\vspace{5pt}

Samenvattend blijkt uit de literatuur dat de keuze van het model, 
de selectie van de features en de contextuele factoren bepalende elementen zijn voor het voorspellen van voetbalwedstrijden. 
Machine learning heeft zijn sterktes, maar het succes ervan hangt sterk af van de kwaliteit van de data, de gekozen variabelen en de insteek van het onderzoek. 
Met deze inzichten gaan we dit onderzoek verder uitdiepen in de Jupiler Pro League.
