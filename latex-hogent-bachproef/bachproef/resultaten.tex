%%=============================================================================
%% Resultaten
%%=============================================================================

\chapter{\IfLanguageName{dutch}{Resultaten}{Results}}%
\label{ch:resultaten}

In dit hoofdstuk worden de resultaten van het empirisch onderzoek gepresenteerd. Het doel is om op een gestructureerde en objectieve manier de bevindingen weer te geven, zonder uitgebreide interpretatie. De interpretatie en kritische reflectie volgen in Hoofdstuk~\ref{ch:discussie}.

De resultaten worden opgesplitst in verschillende onderdelen, overeenkomstig de fasen uit de methodologie.

\section{Beschrijvende data-analyse}

Eerst wordt een overzicht gegeven van de samengestelde dataset. Hierbij worden kernstatistieken besproken, zoals het aantal wedstrijden, seizoenen, variabelen en eventuele ontbrekende waarden. Daarnaast worden verdelingen van de belangrijkste doelvariabelen (zoals eindresultaat, aantal doelpunten en aantal kaarten) gevisualiseerd en kort toegelicht.

\section{Modelprestaties}

Vervolgens worden de prestaties van de geselecteerde machine-learningmodellen gerapporteerd. Per doelvariabele worden relevante evaluatiemetrieken besproken, zoals nauwkeurigheid, F1-score, mean absolute error of andere geschikte maatstaven afhankelijk van het type voorspelling (classificatie of regressie).

De modellen worden onderling vergeleken op basis van:
\begin{itemize}
    \item voorspellende nauwkeurigheid,
    \item stabiliteit over verschillende seizoenen,
    \item mate van overfitting,
    \item generaliseerbaarheid naar niet eerder geobserveerde wedstrijden.
\end{itemize}

\section{Belangrijkste voorspellende factoren}

In dit onderdeel worden de door de modellen geïdentificeerde belangrijkste factoren gepresenteerd. Afhankelijk van het gebruikte model gebeurt dit via coëfficiënten, feature importance-scores of andere interpreteerbare methoden.

De resultaten tonen welke variabelen consistent bijdragen aan de voorspelling van wedstrijdstatistieken en in welke richting deze invloed zich manifesteert.

\section{Vergelijking met expertinschattingen}

Tot slot worden de verzamelde expertinschattingen samengevat. Hierbij wordt gerapporteerd:
\begin{itemize}
    \item welke factoren door experts als doorslaggevend worden beschouwd,
    \item in welke mate experts onderling overeenstemmen,
    \item in welke mate hun voorspellingen overeenkomen met de modelvoorspellingen.
\end{itemize}

Deze vergelijking vormt de basis voor de verdere interpretatie in het volgende hoofdstuk.