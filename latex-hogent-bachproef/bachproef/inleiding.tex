%%=============================================================================
%% Inleiding
%%=============================================================================

\chapter{\IfLanguageName{dutch}{Inleiding}{Introduction}}%
\label{ch:inleiding}

Voetbal is een van de bekendste sporten ter wereld. Jaarlijks worden er miljoenen wedstrijden gespeeld, zowel op professioneel als amateurniveau.
Op hoger niveau wordt er steeds meer gebruik gemaakt van data-analyses voor optimale voorbereidingen en prestaties.
Ook binnen het Belgische voetbal, de \textit{Jupiler Pro League}, groeit de interesse in deze technieken.
Bettingbedrijven, clubs, media en analisten gebruiken steeds meer statistieken om prestaties te duiden en voorspellingen te maken.

\vspace{5pt}

Tegelijkertijd blijft menselijke expertise een centrale rol spelen.
Trainers, analisten en voetbalexperts kijken bij hun voorspellingen niet alleen naar cijfers, 
maar ook naar ervaring, tactisch inzicht en de context, zoals vorm, motivatie en wedstrijdomstandigheden.
De huidige rage rond AI en machine-learningmodellen roept dan ook een relevante vraag op:
In welke mate komen voorspellingen en het inschatten van het belang van bepaalde factoren overeen tussen menselijke experts en data-gedreven modellen?

\vspace{5pt}

Deze bachelorproef combineert sportanalyse, machine learning en interpreteerbare modellen. 
Het onderzoek richt zich op het identificeren van bepalende factoren en het voorspellen van wedstrijdstatistieken in de Jupiler Pro League 
en vergelijkt deze met de inschattingen van voetbalexperts.

\section{\IfLanguageName{dutch}{Probleemstelling}{Problem Statement}}%
\label{sec:probleemstelling}

In het voetbalmilieu worden beslissingen binnen het bestuur en de ploeg vaak genomen op basis van statistische analyse en menselijk inzicht.
Ondanks vele voorspellingsmodellen die gespecialiseerd zijn in sportwedstrijden, 
is het nog steeds niet helemaal duidelijk welke factoren in de Belgische competitie al dan niet invloed hebben.
Bovendien is het verband tussen modelverklaringen en menselijke redeneringen nauwelijks besproken.

\vspace{5pt}

Inzicht in deze verschillen kan een belangrijke meerwaarde bieden voor alle stakeholders binnen het voetbal.
Wanneer modellen andere invloedsfactoren identificeren dan experts doen, 
kan dit aanleiding geven tot het bijsturen van analysemethoden en de beslissingen van trainers.
Daarentegen kan overeenstemming tussen beide wijzen op consistente, onderliggende patronen binnen wedstrijddata.

\vspace{5pt}

Het probleem bestaat dan ook uit twee aspecten:
enerzijds de nood aan bepalende factoren binnen wedstrijdstatistieken van de Jupiler Pro League, met bijhorende voorspellingen, 
anderzijds de duiding van de verschillen en overeenkomsten met de inschattingen van experts.

\section{\IfLanguageName{dutch}{Onderzoeksvraag}{Research question}}%
\label{sec:onderzoeksvraag}

De centrale onderzoeksvraag van deze bachelorproef luidt als volgt:

\vspace{5pt}

\textit{Welke factoren zijn het meest bepalend bij het voorspellen van wedstrijdstatistieken in de Jupiler Pro League, 
en in welke mate verschillen menselijke expertinschattingen van interpreteerbare machine-learningmodellen in hun voorspellingen en verklaringen?}

\vspace{5pt}

Deze onderzoeksvraag wordt opgesplitst in volgende deelvragen:

\textbf{Probleemdomein}
\begin{itemize}
    \item Welke wedstrijdstatistieken (zoals eindresultaat, aantal doelpunten en aantal kaarten) zijn het meest relevant om te voorspellen?
    \item Welke externe en contextuele factoren (zoals ploeg, locatie, tijdstip, vorm, onderlinge resultaten en andere variabelen) hebben potentieel invloed op deze wedstrijdstatistieken?
    \item Welke factoren beschouwen voetbalexperts als doorslaggevend bij het inschatten van wedstrijdverloop en -uitkomsten?
\end{itemize}

\textbf{Oplossingsdomein}
\begin{itemize}
    \item Hoe kunnen relevante factoren worden verzameld, gestructureerd en geanalyseerd met behulp van data-analyse?
    \item Welke machine-learningmodellen zijn het meest geschikt voor het voorspellen van wedstrijdstatistieken in de Jupiler Pro League?
    \item Welke factoren worden door de geselecteerde modellen als belangrijk geïdentificeerd, en hoe verhouden deze zich tot de redenering van experts?
    \item Hoe presteren de geselecteerde modellen in termen van nauwkeurigheid, betrouwbaarheid en generaliseerbaarheid bij het voorspellen van toekomstige wedstrijden?
\end{itemize}

\section{\IfLanguageName{dutch}{Onderzoeksdoelstelling}{Research objective}}%
\label{sec:onderzoeksdoelstelling}

Deze bachelorproef heeft als doel het ontwerpen en evalueren van transparante voorspellingsmodellen voor wedstrijdstatistieken in de Jupiler Pro League. 
Vervolgens wordt een vergelijking gemaakt tussen de factoren die deze modellen als bepalend beschouwen en de oordelen van menselijke experts.

Het beoogde resultaat is een vergelijkende studie waarin:
\begin{itemize}
    \item een database wordt opgezet met verwerkte historische wedstrijdgegevens, inclusief externe factoren zoals weersomstandigheden en stadioncapaciteit;
    \item verschillende interpreteerbare machine-learningmodellen worden ontwikkeld en geëvalueerd;
    \item de belangrijkste voorspellende factoren en bijbehorende modeluitkomsten worden vastgesteld aan de hand van feature-importance-analyses en machine-learningtechnieken;
    \item expertinschattingen worden verzameld en geanalyseerd;
    \item verschillen en overeenkomsten tussen modelverklaringen en menselijke redeneringen worden gedocumenteerd.
\end{itemize}

Als succescriteria gelden: het verkrijgen van reproduceerbare resultaten van het model, 
het aantonen van statistisch significante feature-belangrijkheden en het geven van onderbouwde conclusies 
omtrent de relatie tussen menselijke kennis en modelvoorspellingen.

\section{\IfLanguageName{dutch}{Opzet van deze bachelorproef}{Structure of this bachelor thesis}}%
\label{sec:opzet-bachelorproef}

De rest van deze bachelorproef is als volgt opgebouwd:

In Hoofdstuk~\ref{ch:stand-van-zaken} wordt een overzicht gegeven van de stand van zaken binnen het onderzoeksdomein, 
op basis van een literatuurstudie rond sportvoorspellingen, machine learning en interpreteerbare modellen.

In Hoofdstuk~\ref{ch:methodologie} wordt de methodologie toegelicht en worden de gebruikte onderzoekstechnieken besproken 
om een antwoord te kunnen formuleren op de onderzoeksvragen.

In Hoofdstuk~\ref{ch:dataverzameling} wordt de dataverzameling en voorbereiding van de dataset besproken, waaronder datacleaning, 
feature-engineering en de opdeling in trainings- en testdata.

In Hoofdstuk~\ref{ch:modelontwikkeling} wordt de ontwikkeling en optimalisatie van de machine-learningmodellen toegelicht.

In Hoofdstuk~\ref{ch:resultaten} worden de experimentele resultaten gepresenteerd, inclusief modelprestaties.

In Hoofdstuk~\ref{ch:analyse-discussie} wordt dieper ingegaan op de belangrijkste voorspellende factoren en worden de resultaten vergeleken met expertinschattingen. 
Vervolgens worden deze bevindingen kritisch geïnterpreteerd in functie van de gestelde onderzoeksvragen.

In Hoofdstuk~\ref{ch:conclusie}, tenslotte, wordt de conclusie gegeven en een antwoord geformuleerd op de onderzoeksvragen. 
Daarbij wordt ook een aanzet gegeven voor toekomstig onderzoek binnen dit domein.