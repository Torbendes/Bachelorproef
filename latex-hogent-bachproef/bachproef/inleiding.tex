%%=============================================================================
%% Inleiding
%%=============================================================================

\chapter{\IfLanguageName{dutch}{Inleiding}{Introduction}}%
\label{ch:inleiding}

Voetbal is wereldwijd één van de meest geanalyseerde en besproken sporten. In de professionele competities wordt steeds vaker gebruikgemaakt van data-analyse om prestaties te evalueren, transfers te onderbouwen en wedstrijdvoorbereidingen te optimaliseren. Ook binnen de Belgische \textit{Jupiler Pro League} groeit de interesse in datagedreven besluitvorming. Clubs verzamelen grote hoeveelheden wedstrijd- en spelersdata, terwijl media en analisten statistieken gebruiken om prestaties te duiden en voorspellingen te maken.

Tegelijkertijd blijft menselijke expertise een centrale rol spelen. Trainers, analisten en voetbalexperts baseren hun inschattingen niet alleen op cijfers, maar ook op ervaring, tactisch inzicht en contextuele factoren zoals vorm, motivatie of wedstrijdomstandigheden. De opkomst van machine-learningmodellen roept dan ook een relevante vraag op: in welke mate komen datagedreven voorspellingen overeen met menselijke expertinschattingen, en waar verschillen ze fundamenteel?

Deze bachelorproef situeert zich op het kruispunt van sportanalyse, voorspellende modellering en interpreteerbare artificiële intelligentie. Het onderzoek richt zich op het identificeren van bepalende factoren bij het voorspellen van wedstrijdstatistieken in de Jupiler Pro League en vergelijkt deze met de redeneringen van menselijke experts.

\section{\IfLanguageName{dutch}{Probleemstelling}{Problem Statement}}%
\label{sec:probleemstelling}

Binnen professionele voetbalomgevingen worden beslissingen vaak genomen op basis van een combinatie van statistische analyse en menselijke inschatting. Hoewel er talrijke voorspellingsmodellen bestaan voor sportwedstrijden, is het vaak onduidelijk welke factoren in de Belgische competitie het meest doorslaggevend zijn. Bovendien blijft de relatie tussen modelverklaringen en menselijke redeneringen onderbelicht.

Voor clubs in de Jupiler Pro League, sportanalisten, data-analisten binnen professionele voetbalorganisaties en sportjournalisten kan inzicht in deze verschillen een belangrijke meerwaarde betekenen. Indien blijkt dat modellen systematisch andere factoren als belangrijk identificeren dan experts, kan dit aanleiding geven tot bijsturing van analysemethoden of besluitvormingsprocessen. Omgekeerd kan overeenstemming tussen beide wijzen op robuuste, onderliggende patronen binnen wedstrijddata.

Het probleem dat centraal staat in deze bachelorproef is dan ook tweeledig: enerzijds is er nood aan een gestructureerde identificatie van bepalende voorspellende factoren binnen wedstrijdstatistieken van de Jupiler Pro League, anderzijds is er nood aan inzicht in de mate waarin deze factoren overeenstemmen met menselijke expertinschattingen.

\section{\IfLanguageName{dutch}{Onderzoeksvraag}{Research question}}%
\label{sec:onderzoeksvraag}

De centrale onderzoeksvraag van deze bachelorproef luidt als volgt:

\textit{Welke factoren zijn het meest bepalend bij het voorspellen van wedstrijdstatistieken in de Jupiler Pro League, en in welke mate verschillen menselijke expertinschattingen van interpreteerbare machine-learningmodellen in hun voorspellingen en verklaringen?}

Deze onderzoeksvraag wordt opgesplitst in volgende deelvragen:

\textbf{Probleemdomein}
\begin{itemize}
    \item Welke wedstrijdstatistieken (zoals eindresultaat, aantal doelpunten en aantal kaarten) zijn het meest relevant om te voorspellen?
    \item Welke externe en contextuele factoren (zoals ploeg, locatie, tijdstip, vorm, onderlinge resultaten en andere variabelen) hebben potentieel invloed op deze wedstrijdstatistieken?
    \item Welke factoren beschouwen voetbalexperts als doorslaggevend bij het inschatten van wedstrijdverloop en -uitkomsten?
\end{itemize}

\textbf{Oplossingsdomein}
\begin{itemize}
    \item Hoe kunnen relevante factoren worden verzameld, gestructureerd en geanalyseerd met behulp van data-analyse?
    \item Welke machine-learningmodellen zijn het meest geschikt voor het voorspellen van wedstrijdstatistieken in de Jupiler Pro League?
    \item Welke factoren worden door de geselecteerde modellen als belangrijk geïdentificeerd, en hoe verhouden deze zich tot de redenering van experts?
    \item Hoe presteren de geselecteerde modellen in termen van nauwkeurigheid, betrouwbaarheid en generaliseerbaarheid bij het voorspellen van toekomstige wedstrijden?
\end{itemize}

\section{\IfLanguageName{dutch}{Onderzoeksdoelstelling}{Research objective}}%
\label{sec:onderzoeksdoelstelling}

Het doel van deze bachelorproef is het ontwikkelen en evalueren van interpreteerbare voorspellingsmodellen voor wedstrijdstatistieken binnen de Jupiler Pro League, en het systematisch vergelijken van de geïdentificeerde bepalende factoren met menselijke expertinschattingen.

Het beoogde resultaat is een vergelijkende studie waarin:
\begin{itemize}
    \item een gestructureerde dataset wordt opgebouwd op basis van historische wedstrijdgegevens;
    \item één of meerdere interpreteerbare machine-learningmodellen worden ontwikkeld en geëvalueerd;
    \item de belangrijkste voorspellende factoren worden geïdentificeerd via modelinterpretatie;
    \item expertinschattingen worden verzameld en geanalyseerd;
    \item verschillen en overeenkomsten tussen modelverklaringen en menselijke redeneringen systematisch worden gedocumenteerd.
\end{itemize}

Succescriteria zijn onder meer het behalen van reproduceerbare modelresultaten, het aantonen van statistisch onderbouwde factorbelangrijken, en het formuleren van onderbouwde conclusies omtrent de relatie tussen menselijke expertise en modelgebaseerde voorspellingen.

\section{\IfLanguageName{dutch}{Opzet van deze bachelorproef}{Structure of this bachelor thesis}}%
\label{sec:opzet-bachelorproef}

De rest van deze bachelorproef is als volgt opgebouwd:

In Hoofdstuk~\ref{ch:stand-van-zaken} wordt een overzicht gegeven van de stand van zaken binnen het onderzoeksdomein, op basis van een literatuurstudie rond sportvoorspellingen, machine learning en interpreteerbare modellen.

In Hoofdstuk~\ref{ch:methodologie} wordt de methodologie toegelicht en worden de gebruikte onderzoekstechnieken besproken om een antwoord te kunnen formuleren op de onderzoeksvragen. Hierbij wordt onder meer ingegaan op dataverzameling, feature-engineering, modelselectie en evaluatiemethoden.

In Hoofdstuk~\ref{ch:resultaten} worden de experimentele resultaten gepresenteerd, inclusief modelprestaties en een analyse van de belangrijkste voorspellende factoren, evenals de vergelijking met expertinschattingen.

In Hoofdstuk~\ref{ch:discussie} worden de resultaten kritisch besproken en geïnterpreteerd in functie van de onderzoeksvragen.

In Hoofdstuk~\ref{ch:conclusie}, tenslotte, wordt de conclusie gegeven en een antwoord geformuleerd op de onderzoeksvragen. Daarbij wordt ook een aanzet gegeven voor toekomstig onderzoek binnen dit domein.