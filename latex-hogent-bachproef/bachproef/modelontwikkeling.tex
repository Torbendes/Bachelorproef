\chapter{\IfLanguageName{dutch}{Modelontwikkeling}{Model development}}%
\label{ch:modelontwikkeling}

In dit hoofdstuk wordt de ontwikkeling van de voorspellingsmodellen beschreven. 
Waar het vorige hoofdstuk focuste op data, staat hier de technische uitwerking van de modellen centraal.

\section{Selectie van modellen}

Eerst wordt gemotiveerd welke machine-learningmodellen werden geselecteerd en waarom. Hierbij wordt besproken:
\begin{itemize}
    \item welke algoritmes werden getest,
    \item waarom deze geschikt zijn voor classificatieproblemen,
    \item hoe ze zich theoretisch tot elkaar verhouden.
\end{itemize}

\section{Training en optimalisatie}

Vervolgens wordt toegelicht hoe de modellen werden getraind en geoptimaliseerd. Hierbij wordt ingegaan op:
\begin{itemize}
    \item de gebruikte trainingsstrategie,
    \item hyperparameterafstemming,
    \item maatregelen tegen overfitting.
\end{itemize}

Tot slot wordt besproken hoe interpretatiemethoden werden geïntegreerd in het modelontwikkelingsproces.