\chapter{\IfLanguageName{dutch}{Analyse en discussie}{Analysis and discussion}}%
\label{ch:analyse}

In dit hoofdstuk worden de resultaten geïnterpreteerd in het licht van de onderzoeksvragen. 
Waar in het vorige hoofdstuk de nadruk lag op objectieve rapportering, staat hier de duiding en kritische reflectie centraal.

\section{Interpretatie van modelprestaties}

Eerst wordt besproken in welke mate de ontwikkelde modellen erin slagen om wedstrijduitslagen accuraat te voorspellen. Hierbij wordt nagegaan:
\begin{itemize}
    \item of de behaalde prestaties praktisch relevant zijn,
    \item welke doelvariabelen beter voorspelbaar blijken dan andere,
    \item welke beperkingen zichtbaar worden in termen van generaliseerbaarheid.
\end{itemize}

Daarnaast wordt gereflecteerd op de keuze van de modellen en de impact van feature-engineering op de uiteindelijke prestaties.

\section{Analyse van bepalende factoren}

Vervolgens wordt dieper ingegaan op de factoren die door de modellen als belangrijk werden geïdentificeerd. Er wordt besproken:
\begin{itemize}
    \item welke variabelen consistent een sterke invloed vertonen,
    \item of deze overeenkomen met theoretische verwachtingen,
    \item in welke mate contextuele factoren een significante rol spelen.
\end{itemize}

Tot slot wordt gereflecteerd op de vergelijking tussen menselijke expertise en algoritmische voorspellingen en de implicaties hiervan voor sportanalyse.