\chapter{\IfLanguageName{dutch}{Methodologie}{Methodology}}%
\label{ch:methodologie}

In dit hoofdstuk wordt toegelicht hoe dit onderzoek werd uitgevoerd. 
De methodologie beschrijft de verschillende fasen van het onderzoeksproces, 
de gehanteerde werkwijze en de gemaakte keuzes. 
Het doel van dit hoofdstuk is een helder overzicht te geven van het plan van aanpak, 
zonder reeds in te gaan op concrete resultaten.

\section{Onderzoeksopzet}

Deze bachelorproef hanteert een kwantitatieve onderzoeksaanpak waarbij historische wedstrijddata 
worden gebruikt om machine-learningmodellen te ontwikkelen en te evalueren. 
De centrale doelstelling is het vergelijken van algoritmische voorspellingen met menselijke expertise binnen de context van voetbalwedstrijden.

Het onderzoek werd opgebouwd in vier opeenvolgende fasen:
\begin{itemize}
    \item dataverzameling en voorbereiding,
    \item modelontwikkeling,
    \item evaluatie van modelprestaties,
    \item vergelijking met expertvoorspellingen en interpretatie.
\end{itemize}

Elke fase wordt in een afzonderlijk hoofdstuk verder uitgewerkt.

\section{Dataverzameling en voorbereiding}

In de eerste fase werden historische wedstrijdgegevens verzameld. 
De selectie van variabelen gebeurde op basis van inzichten uit de literatuurstudie, 
waarbij zowel prestatie-indicatoren als contextuele factoren werden meegenomen.

De voorbereiding van de data omvatte onder andere:
\begin{itemize}
    \item het opschonen van de dataset,
    \item het behandelen van ontbrekende waarden,
    \item het transformeren en coderen van variabelen,
    \item het opsplitsen van de data in trainings- en testsets.
\end{itemize}

Deze stappen waren noodzakelijk om betrouwbare en reproduceerbare modeltraining mogelijk te maken.

\section{Modelontwikkeling}

In de tweede fase werden verschillende machine-learningmodellen ontwikkeld. 
De selectie van algoritmes gebeurde op basis van hun geschiktheid voor classificatieproblemen 
en hun frequent gebruik binnen sportanalyse.

Tijdens deze fase werd aandacht besteed aan:
\begin{itemize}
    \item het trainen van de modellen op historische data,
    \item het optimaliseren van hyperparameters,
    \item het beperken van overfitting via geschikte validatiestrategieën.
\end{itemize}

Daarnaast werd voorzien in methoden om modeluitkomsten interpreteerbaar te maken.

\section{Evaluatie van modelprestaties}

De derde fase bestond uit het objectief evalueren van de ontwikkelde modellen. 
Hiervoor werden verschillende evaluatiemetrieken gebruikt die aansluiten bij classificatieproblemen en probabilistische voorspellingen.

Er werd bewust gekozen voor meerdere maatstaven om een vertekend beeld te vermijden en een evenwichtige vergelijking mogelijk te maken.

\section{Vergelijking met menselijke expertise}

In de laatste fase werden de algoritmische voorspellingen vergeleken met menselijke inschattingen. 
Hierbij werd onderzocht in welke mate er verschillen optreden in nauwkeurigheid, consistentie en patroonherkenning.

Deze vergelijking laat toe om niet enkel technische prestaties te analyseren, 
maar ook bredere implicaties te formuleren omtrent de rol van menselijke expertise tegenover datagedreven modellen.

\vspace{5pt}

De hier beschreven methodologische aanpak vormt de basis voor de verdere uitwerking in de volgende hoofdstukken, 
waarin elke fase gedetailleerd wordt besproken.